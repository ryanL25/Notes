\documentclass[a4paper]{article}

\usepackage[utf8]{inputenc}
\usepackage[T1]{fontenc}
\usepackage{amsmath, amssymb}
\usepackage{parskip}
\usepackage{float}

% figure support
\usepackage{import}
\usepackage{xifthen}
\pdfminorversion=7
\usepackage{pdfpages}
\usepackage{transparent}
\newcommand{\incfig}[1]{%
	\def\svgwidth{\columnwidth}
	\import{./figures/}{#1.pdf_tex}
}

\title{160 Lecture 2}
\date{26 Jan 2026} 
\pdfsuppresswarningpagegroup=1

\begin{document}
\maketitle

\section{Unit Variation over an Interval} 
How do we minimize a function over an Interval (a, b]? 

Terminology: 
$$x \in \mathbb{R}$$
\[F: \mathbb{R} \rightarrow \mathbb{R}\]
X in the real numbers, and F is a function mapping from Real numbers to Real numbers

$x$ is the variable or parameter or decision. 

In a convex form, we write it as 

\begin{align*}
	\text{min} \quad  &f(x)\\
	s.t. \quad& x \ in (a,b]
.\end{align*}

We can assume $a, b = \pm \infty$ as well. 

\subsection*{Minima and Maxima}
A point is a global min if $f(x*) \leq f(x) \forall x \in (a, b]$. similarily, global max just flips the sign to $\geq$ 

However, this is often hard to find. Local minimums are easier to find. $f(x*) \leq f(x) \forall I \ \{x\}$,  or in other words $\exists f(x*) \leq f(x) \forall x \in X \cap \{x \pm \Delta\}$. Intuitively, a large $\Delta$ is not going to work here, because it might encompass more local mins. However, it still works because we say there exists. 

A strict local min is where $F(x^*) < f(x) \forall x \in \{x-\delta, x+\delta\}\backslash \{x*\}$. Basically, it is a local min that is not repeated, such as the function $f(x) = 3$. On the interval from [4,5], there is a local min of 3, however, it is not strict because there are infinitely many local mins in the neighborhood. 

These naturally let us ask the following two questions about optimization. 
\begin{itemize}
	\item How to characterize those points?
	\item How to find the numerical values of those points? 
\end{itemize}

To answer these questions we use Taylor Series Approximations or Expansions. 

\subsection*{Taylor Series}

1.) Dominance: The idea that in a taylor series, if $\Delta$ is small, then the upper term polynomial term dominates the rest. How small is small? EH Who cares? The basic idea is ignore high order terms 

Mean Value Theory: 
\begin{enumerate}
	\item First order approximation (Linear) : $F(x + \Delta) = f(x) \sim f(x) + f'(x)\Delta$. There $\exists z$ from $x$ to $x + \Delta$ $f(x + \Delta) = f(x) + f'(z)\Delta$
	\item Second order approximation (quadratic) $f(x + \Delta) \sim f'(x)\Delta + \frac{f''(x + \Delta)}{2}$. Then follows first order approx steps
\end{enumerate}

EX: Imagine $f(x) = e^x$. find the Linear approx of this function. 

So we need to find the approximation around the nominal point. here we will choose $0$.
\[
f(0 + \Delta) = f(0) + f'(z)\Delta\]

\[f(0 + \Delta) = 0 + e^{z}\Delta\]


If $f(x^*)$ is a local min of a function, then $f'(x^*) = 0$. Intuition is picking a neighborhood, zoom in and it slowly becomes flat. Later we will define tangent plan and others.  

Proof: If $F(x*) \neq 0$, then $x*$ is not a min.    

2.) Truncation. 




 
\end{document}
