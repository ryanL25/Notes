
\documentclass{report}

\input{../../preamble}
\input{../../macros}
\input{../../letterfonts}

\title{\Huge{Data140}\ Disc Notes}
\author{\huge{Ryan Lin}}
\date{}

\begin{document}
\maketitle
\newpage% or \cleardoublepage
% \pdfbookmark[<level>]{<title>}{<dest>}
\pdfbookmark[section]{\contentsname}{toc}
\tableofcontents
\pagebreak

\chapter{}
\section{Discussion 02}

\ex{Ch 5 Ex 12}{Small org has n> 12 workers. Assume each worker's birthday is uniformly distributed. Find the chance there is at least one month in which none of the workers has their birthday. 

Let $M_i$ be if month $i$ has no birthdays. 

\[P(\Cup M_i) = \sum p(M_i) - \sum_i \sum_{j = i}^N P(M_i, M_j) + \dots \]
\begin{align*}
	& = \sum P(M_i) - \sum P(M_i M_j) + \dots \sum_k^{12}(-1)^{k+1} 12C(12-k) \\ 
	& = 12P(M_1)^{n} - 12C2 P(M_1, M_2)^{n} + \dots \\
	& = \frac{11}{12}^{n} - \frac{10}{12}^{n}  + \dots
.\end{align*}
}

\ex{Ch5 Ex 9 }{Consider 5 card hand dealth from a standard card deck. find probability of being dealt. 

a.) four of a kind 
\[P(\text{four of a kind}) = \frac{\# outcomes}{\text{total outcomes}} = \frac{13 \cdot 12 \cdot 4 C 4 \cdot 4 C 1}{12 C 5}\]
Make sure to check for distinction. It does matter what 


b.) one pair 
\[P(\text{one pair}) = \frac{13 \cdot 12 C 3 \cdot  4 C 2 \cdot 4(4 C 1)}{52 C 5} \]
In this example, bcd are not distinct. We do not care about order for this. Therefor the 12C3. 

}

\section{Sup 02: Waiting Times, Bionmial/Hypergeomotric/Multinomial}
\ex{Chp 6 Ex 2}{If oyu bet on "red" at roulette, chance of winning is 18/38. They are independent. Suppose you keep betting on red and stop when you have won 6 bets 

There are independent bets, pointing us to a binomial distribution. 

a.) chance you place exactly 10 bets?

\[P(\text{exactly 10 bets}) = \binom{9}{5} \frac{18}{38}^5 \frac{20}{38}^4 \cdot \frac{18}{38}\]

This is $P(\text{exactly 10 bets}) = P(X=5, \text{6th red on 10th bet}) = P(\text{5 reds on 1-9}) \cdot P(\text{Red on 10th})$


b.) What is the chance that you place more than 10 bets? 

Let Y be number of reds in 10 bets. $Y\sim Binom(10, \frac{18}{38})$	

\[P(Y \leq 5) = \sum_{k=0}^5 P(Y(y=k) = \sum \binom{10}{k}\frac{18}{38}^k \frac{28}{38}^{10-k} \]


}
\ex{Ch 6 Ex 10}{Test for disease predicts correct with chance .99. Suppose the test is run on 300 patients, independently


a.) Chance for at least 295 patients the result is correct? Find numeral value.

Let $X$ be number of correct results. $X \sim binom(300, .99)$

\[P(X\ge 295) = \sum_{k=295}^{300} P(X=k) = \sum_{k=295}^{300} \binom{300}{k} (.99^k)(.01)^{300-k}\]

b.) Justify a Poisson approx for the chance in (a) and find the value of the approx. 

In general, if you have a Binomial(n,p) $\approx$ Poisson(np), when n is large and p is small. 

Let $Y$ be the number of patients with incorrect test results (failures) 

\[Y \sim binom(300, 0.01)\]

Now we have a big $N$ and small $p$, therefore $Y \approx Poisson(3)$. This is useful because we know $X+Y = 300$, therefore if we want to find $P(x \ge 295) = P(300-Y \ge 295) = P(Y\le 5)$ 

\[P(Y\le 5) = \sum_{k=0}^5 e^{\frac{-33^k}{k!}} \approx .91608\]


}


\ex{Ch 6 Ex 4}{class consists of 40 freshmen, 60 sophomores, 30 juniors, and 20 seniors. SRS of 10 students

a.) distribution of number of sophomores

Model: without replacement. 

Let $S$ be the number of sophomores in the sample.

\[S \sim Hypergeometric(150,60,10)\]

Hyper geometric has 3 arguments, \# in population, \# of Good, \# sample size.


b.) joint dist of number of freshmen and sophomores

Let $F$ be \# Freshmen in the sample
\[P(F=f, S=s) = \frac{\binom{40}{f} \binom{60}{5} \binom{50}{10-(f+s)}}{\binom{150}{10}}, \quad 0\le f+s \le 10\]

We know that $0 \le f+s \le 10$. 


c.) conditional dist of the number of freshmen in the sample given that there are 4 sophomores.

What we do here is the division rule combining parts a and b.
}

\ex{Ch 6 Ex 11}{Seven dice rolled. Find prob of 

a.) exactly 2 6s 

Let $X$ be the number of 6s rolled. $X \sim binom(7, \frac{1}{6})$

\[P(X=2) = \binom{7}{2} \frac{1}{6}^2 \frac{5}{6}^5\]

b.) two fours, two fives, and 3 sixes

\[P(4,4,5,5,6,6,6) = \frac{1}{6} ^2 \cdot \frac{1}{6}^2 \cdot \frac{1}{6}^3\]

Multiply this by number of ways we can get this combination of numbers. Therefore, multiply the original sequence by $\binom{7}{2} \binom{5}{2} \binom{3}{3}$

c.) three of one face and four of another

\[P(\text{three of one face and four another}) = \binom{7}{3} (\frac{1}{6})^3 (\frac{1}{4})^4 \cdot \binom{6}{1} \cdot \binom{5}{1}\]
It can't be $\binom{6}{2}$ because that would be asking for equal rolls counts, because order matters in the problem. You have to get 3 and 4. 

d.) each face appears

e.) three each of two different faces }

\chapter{}
\section{Discussion 03} 
\ex{ch 7 ex 2}{gambler places 2 different kinds of bets. All bets i.i.d.

	She has $ \frac{1}{n} $ of winning first kind of bet. bets $ n $ times. 
She has chance $ \frac{1}{m} $ of winning the second kind of bet bets $ m $ times. 

Suppose $ m \neq n $ and both are large. 

Let $ T $  be  the total number of bets the gambler wins. Find or apporx dist of $ T  $.


\[T = X + Y\]

Where $ X \sim \Binom(n, \frac{1}{n})$ is number of bets she wins with 1st type and $ Y \sim \Binom(m, \frac{1}{m}) $ is number of bets she wins with 2nd type. Since $ n$ and $ m $ are large, we can approx each using a Poisson Dist. 

\[X \approx Poisson(1), Y \approx Poisson(1)\]

Since Poisson can add up if both r.v are independent poisson.

\[T = X + Y = Poisson(1) + Poisson(1) = Poisson(2)\]

}

\ex{ch 7 ex7}{
Each car that I see has chance .2 of being a hybrid and chance .1 of being electric. Independent of other cars.

a.) I see 15 cars. Find chance i see 3 hybrids, 2 electrics, and 10 others. 

Let $ H $ be \# hybrids, $ E $ \# electric cars, $ O $ other types  

\[H,E,C \sim Multinomial(15, [0.2, 0.1, 0.7])\]
\[P(H=3, E=2, O = 10) = \binom{15}{3} \binom{12}{2} .2^3 .1^2 .7^{10}\]


b.) suppose number of cars is $ \Poisson(15) $. Find the chance i see 3 hybrids, 2 electrics, and 10 others. 

Let $ N \sim \Poisson(15) $ be the number of cars seen. 

If we have a binomial or multinomial r.v., they also become Poisson if $ N \sim \Poisson(\mu)$. This is exact, not a approx. 

\[H \sim \Poisson(.2 \cdot 15 = 3), E \sim \Poisson(.1 \cdot 15 = 1.5), O \sim \Poisson(15 \cdot .7 = 10.5)\]

Because of poissonization, it makes $ H,E,C $ independent. 
\[P(H=3, E=2, O=10) = P(H=3)P(E=2)P(O=10) = e^{-3} \frac{3^3}{3!} \cdot e^{-1.5} \frac{1.5^2}{2!} \cdot e^{-10.5} \frac{10.5^{10}}{10!} \]
}

\ex{ch 7, ex 8}{
Suppose you have $ N $  balls where $ N \sim \Poisson(\lambda) $ dist. each ball is thrown into one of $ m $ bins chosen uniformly at random, independent of all other balls. Find $ P(\text{there is an empty bin}) $ 

\[P(\text{there is an empty bin} = )P(\text{at least 1 bin is empty})\]


We can either try complement or inclusion-exclusion. In this case, both work. 

Let $ N_i $ be the number of balls in bin $i$. 


\begin{align*}
	P(\text{at least 1 bin is empty}) &= 1- P(\text{all bins have at least 1 ball}) \\
	&= 1 - \prod_i^m P(N_i \ge 1) \\
	&= 1 - \prod_i^m 1 - e^{-\frac{\lambda}{m}} \frac{\frac{\lambda}{m}^0}{0!} \\
	&= 1  - (1-e^{-\frac{\lambda}{m}} \frac{\frac{\lambda}{m}^0}{0!})^m \\
\end{align*}

We can change it into a product because $N_i \sim \Poisson(\lambda \cdot \frac{1}{m})$, where all $ N_i $ are i.i.d. 







}
\end{document}
