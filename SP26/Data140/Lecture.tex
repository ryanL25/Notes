\documentclass[a4paper]{article}

\usepackage[utf8]{inputenc}
\usepackage[T1]{fontenc}
\usepackage{textcomp}
\usepackage{amsmath, amssymb}
\usepackage{parskip}
\usepackage{float}

% figure support
\usepackage{import}
\usepackage{xifthen}
\pdfminorversion=7
\usepackage{pdfpages}
\usepackage{transparent}
\newcommand{\incfig}[1]{%
	\def\svgwidth{\columnwidth}
	\import{./figures/}{#1.pdf_tex}
}


\pdfsuppresswarningpagegroup=1
\begin{document}
	
\section{Distribution of a Random Variable}
A Distribution is two connected sets. One is the set of all possible values $\Omega$ and the probabilities of those values. This can result in silly blunders :D. So when faced with random variables, list all possible values. (This will give partial credit!)


\[P(\text{rain today} = 0.65, P(\text{rain tomorrow}) = 0.45\]
$P(\text{rain today and tmr}) = P(\text{rain tmr}) \cdot P(\text{rain tmr} \mid \text{rain today})$

We can only say that this probability is bounded by $\le .45$, because one is a subset of another.  
Similarly, if we are looking at the probability at least one day rains, we have an lower bound of $\ge .65$, because worse case tomorrow is inside today. 



\end{document}
